\chapter{Introduction}\label{chap:baseline_introduction}
This document presents the baseline for the AE5810 Thesis Project of the Space Flight Master at the Faculty of Aerospace Engineering of Delft University of Technology and the SOW-MKI92 Research Project of the Master in Artificial Intelligence at the faculty of Social Sciences of Radboud University. Its purpose is to identify the 2-5 most feasible design options that can be used to determine whether the principle of brain adaptation can be leveraged in neuromorphic space hardware.


The baseline report presents the \acrfull{ffd} and \acrfull{fbd} in \cref{chap:baseline_ffd} and \cref{chap:baseline_fbd} respectively. These function descriptions of the system that is to be designed, is then used to generate the \acrfull{rdt} in \cref{chap:baseline_requirements_discovery_tree}. Next, the resource allocation and budget breakdown presented in \cref{chap:baseline_resource_allocation_budget_breakdown}. This is followed by the technical risk assessment in \cref{chap:baseline_technical_risk_assessment}. From the \acrshort{rdt}, the \acrfull{dot} is generated in \cref{chap:baseline_requirements_discovery_tree}. Contingency management is applied in \cref{chap:baseline_contingency_management}. A market analysis is presented in \cref{chap:baseline_market_analysis}, and the sustainable development strategy is presented in \cref{chap:baseline_sustainable_development_management}. To ensure this work is performed with sufficient quality, the reporting and quality control is presented in \cref{chap:baseline_reporting_and_quality_control}. The baseline is concluded in \cref{chap:baseline_conclusion}.