\section{Introduction}\label{sec:introduction}
Space exploration poses several challenges. Launching material into orbit is still expensive, and once in space, the environment can be harsh due to extreme temperatures, a vacuum, and radiation. In addition, deep space missions have to deal with increasing communication delays with greater distances from Earth. Neuromorphic engineering is a field that can yield applications and use-cases in the domain of space exploration to overcome some of these challenges. The use of spiking neural networks (SNN) in neuromorphic architectures has been shown to greatly increase energy efficiency in Artificial Intelligence (AI) applications \cite{davies_loihi_2018}. Accordingly, neuronmorphic space hardware may be a viable option to increase the autonomy of robotic deep space exploration missions. 

This interest in neuromorphic space application has led to research to assess the impact of radiation on neuromorphic space hardware \cite{cantley_impact_2021,roffe_neutron-induced_2021}. Furthermore, research is being performed to increase the fault-tolerance of neuromorphic hardware \cite{tran_design_2011} and into the radiation robustness of neuromorphic hardware for aerospace applications \cite{vaz_cmos_2020}. %This research sets out to take a bio-inspired approach to increasing the radiation robustness of neuromorphic space hardware.

Since there is a trend of launching small-satellite networks such as Starlink, there is an interest in optimisation algorithms that can be used in these swarms. For example, Qin et al. created a distributed weight-based dominating set clustering algorithm\cite{qin2012weight}.
For this research, a distributed minimum dominating set approximation algorithm is selected, that may be used to approximate the minimum satellites in a swarm that can propagate a message to the entire swarm with a single time step. The distributed nature of the algorithm may allow it to leverage the parallel nature of SNNs.
Next, the existing SNN algorithm is enhanced with an implementation inspired by the principle of brain adaptation on neuromorphic hardware. The SNN algorithm is then tested for radiation robustness in a simulation of space radiation exposure induced SEEs. At the time of writing, Intel did not publish radiation tests results on the Loihi 1 and/or 2 chips, as this may result in export licence limitations with respect to nuclear applications \cite{inrc_meeting}. To overcome this challenge, two simplifications are made for this preliminary research. First, non-SNN component SEEs, such as network initialisation errors caused by SEUs, spike routing errors caused by memory bit flips, etc. are ignored. Secondly, from all possible SNN errors, the research focusses neuron death. A recommendation is included for future work that may remove the need for some of these assumptions. % TODO: verify/ensure recommendation is included.

This paper is structured as follows. First, a brief background on known brain adaptation mechanisms is included in \cref{sec:brain_adaptation}. Next, \cref{sec:methodology} presents the Von Neumann and neuromorphic implementations of the minimum dominating set approximation algorithm, and the brain adaptation implementations. Additionally, it presents the experiment setup and approach to simulate orbit radiation exposure. The simulation results and measured radiation robustness are presented in \cref{sec:results}. \Cref{sec:discussion} interprets the impact of these results, and aims to sketch how well the simulation applies to real radiation exposure conditions of neuromorphic hardware. \Cref{sec:conclusion} summarises the research and indicates whether the chosen implementation of brain adaptation principles is successful at increasing the radiation robustness and at what cost such changes are realised. \Cref{sec:recommendations} provides suggestions on how to build upon this work with future research. % TODO: verify if these claims are valid, adjust if necessary.