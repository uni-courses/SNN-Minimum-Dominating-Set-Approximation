\section{Recommendations \& Future Work}\label{sec:recommendations}
%\textit{Recommendations may be generated based on the listed issues in \cref{sec:discussion}. It is foreseeable that dedicated shielding and or redundancy of critical elements of traditional Von Neumann elements in neuromorphic architectures may be advisable, whilst relying on the brain-inspired implementations in the spiking neural networks. Another recommendation that may flow out of this research may concern the integration of brain adaptation principles into the design of spiking neural network implementations that yield space application functionalities. Furthermore, a recommendation may be generated regarding a proposed shift in the moment of training, from pre-trained on the ground, to (re-)training after radiation exposure.}

The overarching research project proposes to explore additional brain adaptation mechanisms and neural coding strategies. This follow-up work is discussed in \cref{subsec:population_coding} to  \cref{subsec:physical_testing}. 
\subsection{Population Coding}\label{subsec:population_coding}
Starting with the population coding, a population of neurons could be used to represent integer values instead of a single neuron. For example, the \verb+spike_once+ neuron with a synaptic output weight of $x$ could be replaced by $x$ neurons along with $m$ excitatory controller neurons that verify whether each of those neurons is still functional. If part of the population dies, the controller neurons can excite parts of the population to compensate this loss. The $m$ controller neurons could inhibit each other and form a redundancy in the redundancy mechanism.

\subsection{Rate Coding}\label{subsec:rate_coding}
The first round of the algorithm by Alipour et al has also been implemented on the Loihi 2 using a rate-coding approach, where the numbers are represented as a frequency. No radiation damage simulation has yet been performed on this implementation. However, it is expected that spike frequency modulation can be leveraged to mitigate radiation effects.

\subsection{Algorithm Selection}\label{subsec:algorithm_selection}
This work has focussed on a particular optimisation problem, it can be noted that clever algorithm selection (and/or design) for radiation robust SNN may be used to exchange approximation accuracy for robustness. For example, instead of selecting an algorithm that breaks if a single neuron dies, one could consider shortest-path algorithms that automatically yield a longer path that works around the neuron death. Furthermore, selecting applications that are closer to natural brain functionalities, such as event-based vision may facilitate brain adaptation mechanisms at a lower level of redundancy. For example, in some deep neural networks (DNN), neuron death may be a feature instead of a bug, as the retraining phase can in some cases be used to increase the generalisability of the DNN. % TODO: doubt: is this an example analogous to brain adaptation?

\subsection{Physical Testing}\label{subsec:physical testing}
Many of the discussed brain adaptation implementations will fail if the boiler-plate architecture of the SNN suffer from SEEs. This issue can be resolved using fault-tolerance acceptance, redundancy and/or local shielding of boiler-plate architecture components, and by taking boiler-plate SEE propagation mechanisms into account in SNN design. The latter would require physical testing or detailed hardware analysis. Industry partners of the Intel Neuromorphic Research Community, such as ESA, NASA and Raytheon are also working on radiation robustness \cite{inrc_meeting} and may be able to share insight in the more detailed radiation effects on the Loihi 2 without incurring export license limitations.