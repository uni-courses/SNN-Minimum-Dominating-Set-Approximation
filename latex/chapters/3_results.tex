\section{Results}\label{sec:results}
The domains specified in \cref{subsec:results_energy_consumption} to \cref{subsec:results_radiation_robustness} are evaluated as part of the presentation of the results of this experiment. The first three are used to put the radiation robustness in perspective. All four domains are compared to the reference baseline presented in \cref{sec:methodology}.

\subsection{Energy Consumption}\label{subsec:results_energy_consumption}
\textit{Depending on the radiation test method, the energy consumption may be monitored hardwarematically, or estimated softwarematically. This may be challenging in the case of reliance on neuromorphic cloud architectures for testing, as some may not provide the ability to measure and/or report energy consumption. An approximation to the energy consumption could be counting the occurrence of neuron spikes. This is because neuromorphic architectures are considered to only consume significant amounts of energy when they spike.}

\subsection{Neuronal \& Synaptic Overcapacity}\label{subsec:results_neuronal_synaptic_overcapacity}
By taking this factor into account, a context can be provided for the added value of using brain-inspired radiation robustness implementations in neuromorphic hardware.

\subsection{Traditional Hardware Element Performance}\label{subsec:results_traditional_hardware_element_performance}
\textit{Typical neuromorphic hardware still contains some traditional/Von Neumann architecture components, such as a memory bus and processor. These are typically used to communicate with the outside world, and to orchestrate/set up/initialise the neural network \cite{TODO}. Given this position high in the functional hierarchy, they may form a critical bottleneck in terms of radiation robustness. To overcome this issue, a recommendation could be included to increase the shielding and/or redundancy of these Von Neumann components, whilst saving mass and redundancy in the spiking neural networks using brain-inspired implementations \cite{johan_kwisthout_personal_corrospondence_feb_15}.}

\subsection{Radiation Robustness}\label{subsec:results_radiation_robustness}
\textit{This section will be used to present the radiation robustness of the tested neuromorphic architectures with- and without brain-inspired implementation.}