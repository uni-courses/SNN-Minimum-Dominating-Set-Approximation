\section{Brain Adaptation}\label{sec:brain_adaptation}
Four methods of brain adaptation are considered within this research; redundancy, reorganisation, niche construction and timing of developmental trajectories.  Starting with redundancy, it is noted that higher vertebrate brains often have multiple neural pathways that can support certain behaviour \cite{johnson_brain_2015}. Using multiple pathways is considered a viable strategy of implementing brain adaptation in the generated SNN. However, for completeness, reorganisation and niche construction are also evaluated. Continuing with reorganisation, evidence suggests that the human brain becomes increasingly hierarchical during post-natal development \cite{supekar_brain_2013}. No levels of hierarchy are found in the generated SNN. It is noted that for more advanced applications, such as on-site learning/retraining, the increases in hierarchy of the SNN might allow for automatic recovery after radiation exposure. An example could be a pre-trained network performing re-training after passing through the Van Allen belts before arriving at another solar body. Next, niche construction has been seen in some organisms that can change the neural pathways to select environment information based on a combination of which information the organism needs and what the brain can best process \cite{johnson_brain_2015}. It is expected that including the aspect of what the brain can best process requires a highly advanced combination of objective functions and information input streams. This option is not considered feasible within this research project. Timing of developmental trajectories can be seen as a compensation mechanism that delays the development of certain brain regions such that the brain can sample information from early environments such that it can optimise its development structure \cite{johnson_brain_2015}. The timing of developmental trajectories could be applied to a Mars rover that learns object detection on Mars instead of on Earth using two different sensory inputs to provide input data and labels. If there are significant learnable differences with respect to the Earth environment, it could lead to a better trained model. In the context of the radiation robustness, the optimisation of the development structure could be used to ignore damaged neurons after radiation exposure. Redundancy is selected as the primary method of brain-adaptation for the selected SNN implementation of the MDS approximation algorithm because it does not require structural hierarchies nor model training.

To see how this redundancy can be implemented the following neural coding mechanisms are evaluated:
\begin{itemize}
    \item \textit{Rate/frequency coding} - Assumes frequency or rate of action potential increases are accompanied by stimulus intensity increases.
    \item \textit{Temporal coding} - Uses high-frequency firing rate fluctuations to convey information. %Some natural frequency oscillations may be used to intelligently restructure the network using this concept.}
    \item \textit{Population coding} - Represents stimuli using combined activities of multiple neurons.% Work by fellow student Fabian Schneider at the Donders Institute showed this method in particular proved useful in the context of robustness SNNs. Fabian's work will be taken into account in the baseline, midterm and final phase where relevant.}
    \item \textit{Sparse coding} - Uses small subsets of neurons to encode items. %Perhaps this mechanism can be used at times to represent redundant information.}
\end{itemize}
Rate/frequency adjustments can be used to increase or lower the \textit{precision} % TODO: verify this is the correct word. 
of the spiking representation of numbers. By increasing the frequency, the relative impact of radiation induced spike omission could be reduced. % TODO: include proof of concept.
Similarly, with population coding, the population size adjustments can be used to lower the relative impact of radiation induced neuron deaths on numerical representation accuracies. No useful implementations for temporal coding and sparse coding are found in the context of radiation robustness of the generated SNN algorithm for the MDS approximation.

In order to maximise the energy efficiency of the designed SNN implementation, no particular neural coding scheme has been used. Instead, some neurons pass a single spike based on input, whereas others spike continuously depending on their input.