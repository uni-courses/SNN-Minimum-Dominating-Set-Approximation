\begin{abstract}
  %Energy efficient applications of artificial intelligence (AI) may be able to increase space robot autonomy. 
  This research sets out to test whether principles of brain adaptation can be leveraged to increase the radiation resistance of neuromorphic space hardware. Neuromorphic architectures provide energy efficient platforms for AI applications in space. Structural similarities between neuromorphic architectures and the brain may allow them to benefit from brain-inspired design on the topic of damage recovery. Space environments can provide challenging conditions with significant radiation exposure, that may damage neuromorphic space hardware. To explore approaches to mitigate such damages, a simulated radiation test with- and without an implementation of brain adaptation is investigated.
  
 The differences in radiation robustness are then analysed and discussed in the context of space applications of neuromorphic hardware. The spiking neural network (SNN) implementation by Diehl et al. of the minimum dominating set approximation algorithm by Alipour et al. is enhanced with brain adaptation mechanisms for these tests. The tests are performed using Intel's Lava 0.3.0 Framework.
\end{abstract}