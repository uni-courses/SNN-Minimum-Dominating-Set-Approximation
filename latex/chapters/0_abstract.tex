\begin{abstract}
  Energy efficient applications of artificial intelligence (AI) may be able to increase space robot autonomy. Neuromorphic architectures provide energy efficient platforms for AI applications in space. These brain-inspired architectures mimic the neuronal and synaptic structure of the brain using spiking neural networks (SNNs). These structural similarities with the brain, may allow them to benefit from brain-inspired design on the topic of damage recovery. Therefore, this research sets out to test whether principles of brain adaptation can be leveraged to increase the radiation resistance of neuromorphic space hardware. This concept is investigated using radiation robustness tests on neuromorphic architecture with- and without brain adaptation implementation. The difference in radiation robustness are then analysed and discussed to put the radiation robustness difference into perspective in the context of space applications of neuromorphic hardware. An SNN implementation of the minimum dominating set approximation algorithm by Alipour et al is created and used for these tests \cite{alipour}. \textit{Simulated radiation robustness is generated by means of an external controller that manages SNN re-configuration, as well as through SNN adaptivity itself.}
\end{abstract}