\section{Discussion}\label{sec:discussion}
%The discussion is used to put the/any radiation robustness differences between using a brain-inspired implementation and regular redundancy in neuromorphic architectures, into perspective. This perspective is generated by discussing the architecture comparison in terms of the characteristics described in \cref{subsec:discussion_performance_trade_off} to \cref{subsec:discussion_space_application_representation_accuracy}.

\subsection{Performance Trade-Off}\label{subsec:discussion_performance_trade_off}
%\textit{Increasing the radiation resistance of a default neuromorphic implementation requires some effort. In this study, this effort is realised in the form of a brain-inspired implementation. Such an implementation comes at the cost of resources. In this case those resources are neurons, synapses and energy consumption. Since the comparison is made in terms of radiation robustness, it is important to determine whether it is not merely the consumption of extra resources that led to performance differences. In particular, for space applications, it is important to determine whether the resources that are consumed, for example the energy consumption, are worth the benefits. Each space mission will have its own trade-off in these terms, however, this subsection can shed some light on the trade-off costs. In essence a quantitative insight can be given in terms of performance enhancement and energy cost increases. Similarly, a quantitative insight can be provided in terms of performance enhancement, and the increase in required neurons/synapses and/or hardware mass (if the increase in neurons/synapses require a larger chip).}

\subsection{Simulation vs Radiation Testing}
%\textit{If the radiation testing is performed softwarematically in this article submission, this section can be used to convey how portable these results are expected to be to practical radiation tests.}

\subsection{Radiation Robustness of Traditional Hardware Elements}
%\textit{The potential bottleneck of traditional Von Neumann hardware elements in neuromorphic hardware, as introduced in \cref{subsec:results_traditional_hardware_element_performance} should be discussed in this section.}

\subsection{Space Application Representation Accuracy}\label{subsec:discussion_space_application_representation_accuracy}
%\textit{Some context can be provided in this section to indicate how/up to what extent the radiation testing that is performed for this article submission applies to real space applications.}